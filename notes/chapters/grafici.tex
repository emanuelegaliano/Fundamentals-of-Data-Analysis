\chapter{Grafici}
I grafici sono strumenti fondamentali per visualizzare e comprendere i dati. Essi permettono di rappresentare informazioni in modo visivo, facilitando l'interpretazione e l'analisi dei dati stessi.

Quando si crea un grafico, la domanda principale da porsi è "Cosa deve rappresentare questo grafico?". La scelta del tipo di grafico dipende dal tipo di dati che si desidera visualizzare e dal messaggio che si vuole comunicare. Alcune domande comuni possono essere:
\begin{itemize}
    \item Voglio mostrare la distribuzione di una singola variabile?
    \item Voglio confrontare più gruppi o categorie?
    \item Voglio analizzare la relazione tra due variabili?
    \item Voglio rappresentare dati temporali o sequenziali?
\end{itemize}

\noindent
Dopo essersi posti queste domande, si sceglie il tipo di grafico più adatto. Di seguito sono elencati alcuni dei tipi di grafici più comuni e le loro caratteristiche principali.

\section{Istogramma}\label{sec:istogramma}
Un Istogramma è un tipo di grafico che rappresenta la distribuzione di un insieme di dati suddividendoli in intervalli (o "bin") e mostrando la frequenza di dati in ciascun intervallo mediante barre verticali. Gli istogrammi sono utilizzati solo per variabili quantitative (numeriche, misurabili) come per esempio l'altezza delle persone, l'età; non avrebbe infatti senso fare un istogramma per una variabile puramente categoriale come il colore degli occhi o il genere poiché l'istogramma rappresenta frequenze lungo un asse numerico continuo.

\begin{figure}[htbp]
    \centering
    \includegraphics[width=0.8  \textwidth]{images/istogramma_altezze.png}
    \caption{Distribuzione simulata delle altezze. Le barre mostrano la frequenza relativa (densità) dei valori osservati; la linea rossa tratteggiata indica la media campionaria.}
    \label{fig:istogramma_altezze}
\end{figure}

\paragraph{Esempio.} 
Consideriamo un insieme di dati che rappresentano le altezze (in cm) di un gruppo di persone. Creiamo un istogramma per visualizzare la distribuzione delle altezze, come mostrato nella figura \ref{fig:istogramma_altezze}. Dalla figura si può evincere facilmente che la maggior parte delle altezze si concentra intorno alla media (circa 170 cm), con una distribuzione che sembra approssimare una curva normale (a campana). Inoltre, l'istogramma mostra che ci sono poche persone con altezze estreme (molto basse o molto alte). 

Dalla sotto-sezione \ref{subsec:pdf} sappiamo che spesso gli istogrammi sono utilizzati per stimare la funzione di densità di probabilità (PDF) di una variabile casuale continua. In questo caso, l'istogramma fornisce una rappresentazione visiva della distribuzione dei dati, permettendo di osservare la forma della distribuzione, la presenza di picchi (modi), la simmetria o asimmetria, e altre caratteristiche importanti.

\begin{figure}[htbp]
    \centering
    \includegraphics[width=.8\textwidth]{images/istogramma_altezze_pdf_normale.png}
    \caption{Istogramma delle altezze con sovrapposta la densità normale stimata dai dati. Le barre mostrano la distribuzione empirica, la curva rappresenta la PDF gaussiana stimata (media e deviazione standard campionarie) e la linea tratteggiata indica la media del campione.}
    \label{fig:istogramma_densita}
\end{figure}

Da questa figura si può osservare come grazie a un istogramma normalizzato sia possibile confrontare la distribuzione empirica dei dati con una distribuzione teorica (in questo caso, la distribuzione normale). Questo confronto può aiutare a valutare quanto bene i dati seguano una determinata distribuzione teorica, fornendo informazioni utili per l'analisi statistica e la modellizzazione.

\section{Grafico a barre}\label{sec:barplot}
Un grafico a barre è un tipo di grafico utilizzato per rappresentare dati categoriali mediante barre rettangolari. Ogni barra rappresenta una categoria e la lunghezza o l'altezza della barra è proporzionale alla frequenza o alla quantità associata a quella categoria.

I grafici a barre sono pricipalmente utilizzati per confrontare valori tra diverse categorie. Questi valori possono essere frequenze assolute, frequenze relative (percentuali) o altre misure quantitative.

\paragraph{Esempio.} 
Consideriamo un insieme di dati che rappresentano il numero di studenti iscritti a diversi corsi universitari in un semestre. Creiamo un grafico a barre per visualizzare il numero di studenti in ciascun corso, come mostrato nella figura \ref{fig:barplot_corsi}.

\begin{figure}[htbp]
    \centering
    \includegraphics[width=\textwidth]{images/barplot_corsi.png}
    \caption{Numero di studenti iscritti a diversi corsi universitari in un semestre. Le barre rappresentano il numero di studenti per ciascun corso.}
    \label{fig:barplot_corsi}
\end{figure}

Dalla figura si può facilmente confrontare il numero di studenti iscritti a ciascun corso. Ad esempio, si nota che il corso di "Informatica" ha il maggior numero di iscritti, mentre il corso di "Biologia" ha il minor numero di iscritti. Questo tipo di grafico è utile per identificare rapidamente le differenze tra le categorie e per prendere decisioni basate sui dati.

\section{Boxplot}\label{sec:boxplot}
Un boxplot è un tipo di grafico utilizzato per rappresentare la distribuzione di un insieme di dati quantitativi attraverso cinque statistiche riassuntive: il minimo, il primo quartile (Q1), la mediana (Q2), il terzo quartile (Q3) e il massimo. I boxplot sono utili per visualizzare la dispersione, la simmetria e la presenza di valori anomali (outlier) nei dati. È particolarmente utile perché consente di individuare facilmente, una volta imparato a leggerlo, caratteristiche importanti della distribuzione dei dati.

\paragraph{Esempio.} 
Consideriamo un insieme di dati che rappresentano i punteggi ottenuti da studenti in un esame. Creiamo un boxplot per visualizzare la distribuzione dei punteggi, come mostrato nella figura \ref{fig:boxplot_punteggi}.

\begin{figure}
    \centering
    \includegraphics[width=0.65\textwidth]{images/boxplot_punteggi_studenti.png}
    \caption{Boxplot dei punteggi ottenuti dagli studenti in un esame. Il boxplot mostra la mediana, i quartili, i valori minimi e massimi, e gli outlier.}
    \label{fig:boxplot_punteggi}
\end{figure}

Dalla figura si può osservare che la mediana dei punteggi è intorno a 20, quindi significa che il 50\% degli studenti ha ottenuto un punteggio inferiore a 20 e il restante 50\% ha ottenuto un punteggio superiore a 20. Inoltre, il boxplot mostra che la maggior parte dei punteggi si concentra tra il primo quartile (circa 15) e il terzo quartile (circa 23). Da questo si potrebbe calcolare l'intervallo interquantile $IQR = 23 - 15 = 8$. Si noti che il minimo è circa 6, il massimo è 31 (30L) mentre è presente un outlier sotto il valore di 5 (indicato con un cerchio). Questo outlier indica che c'è uno studente che ha ottenuto un punteggio significativamente più basso rispetto agli altri.

\section{Grafici di dispersione}\label{sec:scatterplot}
Un grafico di dispersione (scatter plot) è un tipo di grafico utilizzato per visualizzare la relazione tra due variabili quantitative. In un grafico di dispersione, ogni punto rappresenta un'osservazione del dataset, con la posizione orizzontale (asse x) che rappresenta il valore di una variabile e la posizione verticale (asse y) che rappresenta il valore dell'altra variabile. I grafici di dispersione sono utili per identificare correlazioni, tendenze e modelli nei dati durante \textbf{}{l'analisi multivariata}.

\paragraph{Esempio.}
Ipotizziamo di dover analizzare la relazione tra il numero di ore di studio settimanali e i punteggi ottenuti dagli studenti (200 in totale) in un esame (da 0 a 100). Creiamo un grafico di dispersione per visualizzare questa relazione, come mostrato nella figura \ref{fig:scatterplot_studio_punteggio}.

\begin{figure}[htbp]
    \centering
    \includegraphics[width=\textwidth]{images/scatterplot_studio_punteggio.png}
    \caption{Grafico di dispersione che mostra la relazione tra il numero di ore di studio settimanali e i punteggi ottenuti dagli studenti in un esame.}
    \label{fig:scatterplot_studio_punteggio}
\end{figure}

Dalla figura si evince molto facilmente, che i punteggi tendono ad aumentare con l'aumentare delle ore di studio settimanali. Questo suggerisce una correlazione positiva tra le due variabili, indicando che gli studenti che dedicano più tempo allo studio tendono a ottenere punteggi più alti negli esami. Inoltre, si possono notare alcune variazioni nei punteggi per un dato numero di ore di studio, il che indica che altri fattori potrebbero influenzare i risultati degli studenti.

Ci possiamo facilmente convincere di questa cosa calcolando il coefficiente di correlazione di Pearson tra le due variabili:
\[
p = \frac{\mathrm{cov}(X, Y)}{\sigma_X \sigma_Y} \approx 0.99
\]

Da questo calcolo confermiamo la forte correlazione positiva tra le due variabili, come suggerito dal grafico di dispersione.

\subsection{Matrice di scatter plot}\label{subsec:scattermatrix}
Una matrice di scatter plot (o scatter matrix) è una rappresentazione grafica che mostra i grafici di dispersione per tutte le coppie di variabili in un dataset multivariato. Ogni cella della matrice contiene un grafico di dispersione che rappresenta la relazione tra due variabili specifiche, mentre le diagonali possono contenere istogrammi o grafici a densità per ciascuna variabile.

\paragraph{Esempio.}
Ipotizziamo di continuare l'esempio precedente: abbiamo un dataset con più variabili relative agli studenti, come il numero di ore di studio settimanali, i punteggi ottenuti negli esami, il numero di assenze, il tempo dedicato ad attività extracurriculari e il livello di stress. Creiamo una matrice di scatter plot per visualizzare le relazioni tra tutte queste variabili, come mostrato nella figura \ref{fig:scattermatrix_studenti}.

\begin{figure}[htbp]
    \centering
    \includegraphics[width=0.8\textwidth]{images/scatter_matrix_studenti.png}
    \caption{Matrice di scatter plot che mostra le relazioni tra diverse variabili relative agli studenti. Ogni cella contiene un grafico di dispersione per una coppia di variabili, mentre le diagonali mostrano istogrammi delle singole variabili.}
    \label{fig:scattermatrix_studenti}
\end{figure}

Dalla figura si possono osservare diverse relazioni tra le variabili. Ad esempio, si nota una correlazione positiva tra il numero di ore di studio settimanali e i punteggi ottenuti negli esami, come già discusso in precedenza. Inoltre, si può osservare una correlazione negativa tra il numero di assenze e i punteggi degli esami, suggerendo che gli studenti che frequentano regolarmente le lezioni tendono a ottenere risultati migliori. La matrice di scatter plot consente di identificare rapidamente queste relazioni e di individuare eventuali pattern o tendenze nei dati.

\subsection{Scatter plot con intervalli di confidenza}\label{subsec:scatter_ci}
Uno scatter plot con intervalli di confidenza è un tipo di grafico di dispersione che include linee o bande che rappresentano gli intervalli di confidenza attorno a una stima della relazione tra le due variabili. Questi intervalli forniscono una misura della precisione della stima e aiutano a visualizzare l'incertezza associata alla relazione tra le variabili. Sono molto utili per valutare la significatività statistica della relazione osservata, in particolare quando si esegue una regressione lineare.

\paragraph{Esempio.}
Consideriamo nuovamente l'esempio del numero di ore di studio settimanali e dei punteggi ottenuti dagli studenti in un esame. Possiamo stimare una retta di regressione lineare che descriva, in media, come varia il punteggio al variare delle ore di studio e rappresentare graficamente sia tale retta sia l'incertezza sulla stima. In figura \ref{fig:scatterplot_ci_studio_punteggio} è riportato uno scatter plot in cui ogni punto rappresenta uno studente, la linea centrale rappresenta la retta di regressione stimata e la banda ombreggiata attorno alla linea corrisponde all'intervallo di confidenza (unicamente per fini didattici l'intervallo sarà al 99\% per mostrare in modo netto il grafico) della media condizionale del punteggio dato un certo numero di ore di studio.

\begin{figure}[htbp]
    \centering
    \includegraphics[width=0.8\textwidth]{images/scatterplot_ci_studio_punteggio.png}
    \caption{Scatter plot che mostra la relazione tra ore di studio settimanali e punteggio d'esame, con sovrapposta la retta di regressione lineare stimata e la banda di intervallo di confidenza al 99\%.}
    \label{fig:scatterplot_ci_studio_punteggio}
\end{figure}

Dalla figura si osserva che, in media, all'aumentare delle ore di studio il punteggio tende ad aumentare in modo quasi lineare. L'intervallo di confidenza è più stretto nella zona in cui si concentrano la maggior parte delle osservazioni (valori di ore di studio più frequenti) e tende ad allargarsi agli estremi, dove i dati sono più scarsi. Questo tipo di rappresentazione permette non solo di vedere la tendenza generale (la retta di regressione), ma anche di valutare l'incertezza associata alla stima di tale tendenza.


\section{Hexbin plot}\label{sec:hexbinplot}
Un hexbin plot è un tipo di grafico utilizzato per visualizzare la densità di punti in un grafico di dispersione bidimensionale. Invece di rappresentare ogni punto individualmente, l'hexbin plot suddivide l'area del grafico in celle esagonali (hexagons) e conta il numero di punti che cadono in ciascuna cella. La densità dei punti in ogni cella viene quindi rappresentata mediante una scala di colori, con colori più scuri che indicano una maggiore densità di punti.

Viene considerato l'istogramma bidimensionale con celle esagonali, ed è particolarmente utile quando si lavora con grandi quantità di dati, poiché riduce il sovraffollamento e rende più facile identificare le aree di alta densità.

\paragraph{Esempio.}
Consideriamo un insieme di dati che rappresentano le altezze e i pesi di un gruppo di persone. Creiamo un hexbin plot per visualizzare la densità dei punti in relazione a queste due variabili, come mostrato nella figura \ref{fig:hexbin_altezza_peso}.

\begin{figure}[htbp]
    \centering
    \includegraphics[width=0.7\textwidth]{images/hexbin_altezza_peso.png}
    \caption{Hexbin plot che mostra la densità delle osservazioni in relazione all'altezza e al peso di un gruppo di persone. Le celle esagonali rappresentano la densità dei punti, con colori più scuri che indicano una maggiore densità.}
    \label{fig:hexbin_altezza_peso}
\end{figure}

Dalla figura si può osservare che la maggior parte delle persone si concentra in un'area specifica del grafico, indicando una relazione tra altezza e peso. Le celle esagonali più scure rappresentano le combinazioni di altezza e peso più comuni nel dataset, mentre le celle più chiare indicano combinazioni meno frequenti. Anche questo grafico serve nell'analisi multivariata ed è utile per identificare pattern e tendenze nei dati.

\section{Grafici di densità e di contorno}\label{sec:densityplot}
Un grafico di densità (density plot) è una rappresentazione grafica della distribuzione di una variabile continua. A differenza di un istogramma, che suddivide i dati in intervalli discreti, un grafico di densità utilizza una funzione di densità stimata per mostrare la distribuzione dei dati in modo più fluido e continuo. Questo tipo di grafico è utile per visualizzare la forma della distribuzione, identificare picchi (modi) e confrontare più distribuzioni.

Un grafico di contorno (contour plot) è una rappresentazione grafica che mostra le linee di livello (contorni) di una funzione di due variabili. In un grafico di contorno, le linee collegano punti con lo stesso valore della funzione, permettendo di visualizzare la topografia della funzione in uno spazio bidimensionale. I grafici di contorno sono utili per analizzare la relazione tra due variabili e per identificare aree di interesse, come massimi, minimi e punti di sella.

\paragraph{Esempio.}
Ipotizziamo di guardare la distribuzione delle altezze e del peso di un gruppo di persone. Creiamo un grafico di densità bidimensionale e un grafico di contorno per visualizzare la distribuzione congiunta di queste due variabili, come mostrato nella figura \ref{fig:densityplot_altezza_peso}.

\begin{figure}[htbp]
    \centering
    \includegraphics[width=\textwidth]{images/density_contour_altezza_peso.png}
    \caption{Grafico di densità bidimensionale e grafico di contorno che mostrano la distribuzione congiunta di altezza e peso in un gruppo di persone. Le aree più scure nel grafico di densità indicano una maggiore concentrazione di punti, mentre le linee nel grafico di contorno rappresentano i livelli di densità.}
    \label{fig:densityplot_altezza_peso}
\end{figure}

Come si evince dalla figura, il grafico di densità mostra che la maggior parte delle persone si concentra in un'area specifica del grafico, indicando una relazione tra altezza e peso. Le aree più scure rappresentano le combinazioni di altezza e peso più comuni nel dataset. Il grafico di contorno, invece, fornisce una rappresentazione visiva delle linee di livello della densità, permettendo di identificare facilmente le aree di alta e bassa densità.

Questi grafici sono ulteriori strumenti utili nell'analisi multivariata per comprendere meglio la distribuzione e la relazione tra due variabili quantitative.

\section{Heatmaps}\label{sec:heatmaps}
Dalle matrici di correlazione è possibile creare delle heatmaps (mappe di calore) per visualizzare le correlazioni tra più variabili in un dataset. Una heatmap utilizza una scala di colori per rappresentare i valori di una matrice, dove ogni cella della matrice corrisponde a una coppia di variabili e il colore della cella indica la forza e la direzione della correlazione tra quelle variabili.

\paragraph{Esempio.}
Riprendendo l'esempio della matrice di scatter plot nella sotto-sezione \ref{subsec:scattermatrix}, possiamo calcolare la matrice di correlazione tra le variabili relative agli studenti e creare una heatmap per visualizzare queste correlazioni, come mostrato nella figura \ref{fig:heatmap_correlazioni_studenti}.

\begin{figure}[htbp]
    \centering
    \includegraphics[width=0.7\textwidth]{images/heatmap_correlazioni_studenti.png}
    \caption{Heatmap che mostra la matrice di correlazione tra diverse variabili relative agli studenti. I colori indicano la forza e la direzione della correlazione, con colori più scuri che rappresentano correlazioni più forti.}
    \label{fig:heatmap_correlazioni_studenti}
\end{figure}

Dalla figura si può osservare facilmente la forza e la direzione delle correlazioni tra le variabili. Ad esempio, si nota una forte correlazione positiva tra il numero di ore di studio settimanali e i punteggi ottenuti negli esami, come già discusso in precedenza. Inoltre, si può osservare una correlazione negativa tra il numero di assenze e i punteggi degli esami. La heatmap consente di identificare rapidamente queste relazioni e di individuare eventuali pattern o tendenze nei dati, facilitando l'analisi multivariata.

\section{Load plots}\label{sec:loadplots}
Un load plot è un tipo di grafico utilizzato per visualizzare i carichi (loadings) delle variabili originali su componenti principali o fattori in un'analisi di componenti principali (PCA) o in un'analisi fattoriale. I loadings rappresentano l'importanza relativa di ciascuna variabile nella formazione delle componenti principali o dei fattori. Un load plot mostra questi carichi in modo visivo, permettendo di identificare quali variabili contribuiscono maggiormente a ciascuna componente o fattore.

\noindent
In un loadplot:
\begin{itemize}
    \item Le variabili che si "raggruppano" insieme in una direzione simile indicano una forte correlazione tra di esse.
    \item Le variabili che si trovano lontano dall'origine del grafico hanno un maggiore contributo alla componente principale o al fattore rappresentato.
    \item Le variabili che si trovano in direzioni opposte indicano una correlazione negativa tra di esse.
\end{itemize}

\begin{figure}[htbp]
    \centering
    \includegraphics[width=1\textwidth]{images/load_plot_example.png}
    \caption{Load plot della PCA per il dataset Iris. Le frecce rappresentano i loadings delle feature originali sulle prime due componenti principali, indicando il contributo e l’orientamento di ciascuna variabile rispetto a PC1 e PC2.}
    \label{fig:load_plot_example}
\end{figure}