\chapter*{Prefazione}

\setlength{\parskip}{0.8em}
\setlength{\parindent}{0pt}

Queste dispense sono nate come appunti personali per il corso di \emph{Fondamenti di Analisi Dati} tenuto presso il Dipartimento di Matematica e Informatica dell'Universit\`a di Catania tenuto dal Prof. Antonino Furnari nell'anno accademico 2025/2026. L'obiettivo principale di questo materiale è stato fornire una risorsa personale di studio sulla parte teorica del corso, raggruppando concetti chiave, definizioni, teoremi, immagini ed esempi in unico documento. 

Questo file non deve essere visto come un testo ufficiale o completo sull'argomento, in quanto potrebbero esserci errori, omissioni o imprecisioni. Invito pertanto chi legge questo documento a consultare le dispense ufficiali del corso proposte dal docente ed eventuali testi di riferimento consigliati. Inoltre, invito chiunque noti errori o abbia suggerimenti a contattarmi via email:
\begin{itemize}
    \item \textbf{Email personale}: \href{mailto:galianoo.emanuele@gmail.com}{galianoo.emanuele@gmail.com},
    \item \textbf{Email universitaria}: \href{mailto:emanuele.galiano@studium.unict.it}{emanuele.galiano@studium.unict.it};
\end{itemize}

\noindent
oppure se ancora presente online, aprire una \textbf{issue} o una \textbf{pull request} nel repository GitHub associato a queste dispense: 
\begin{center}
\begin{minipage}{0.92\textwidth}
\centering
\url{https://github.com/emanuelegaliano/Fundamentals-of-Data-Analysis}
\end{minipage}
\end{center}

In particolare, all'interno del repository GitHub sono presenti il codice open-source dei sorgenti \LaTeX{} utilizzati per generare queste dispense, oltre a tutti i file di supporto come immagini e codici per la generazione di alcuni grafici presenti nel testo. 